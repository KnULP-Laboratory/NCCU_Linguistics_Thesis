\chapter{Introduction} %章節的名稱
\label{c:intro} %這個章節的標籤,在文內引用時會用到

This is a template of thesis for the Graduate Institute of Linguistics in NCCU. The original version of this template was provided by \href{https://github.com/Walker088/nccu-thesis}{walker088}. To start editing please click on the introduction.tex file in the folder named chapters/. Please follow the instructions after the percentage symbols to complete your thesis. \par % 換段落的時候要加上\par才會換段落喔

這是你的前言。
這是你的前言。\\ % 不然就會像這樣
這是你的前言。\par % 或這樣

% 也就是說,如果你單純只是想換行但沒有要開啟新的段落,可以用兩條反斜線。但如果你是想換新的段落的話,請用\par。

In the environment of \LaTeX \ , if you want to type quotation marks, ``please" follow this `instruction'.
% 在LATEX裡面打左邊的引號要用`(通常在esc下面),雙引號就打兩次。右邊的部分則是跟一般文件一樣以'跟"表示。

As for some symbols that might be included in your thesis, such as \%, \_, and \{\}, please follow this instruction.
% 某些特殊符號是需要在前面加上反斜線才會出現的。

Finally, you can change the format of your font by adding some commands to them. For example, \textbf{bold faced words}, \textit{italicized words}, and \underline{underlined words}. \textbf{\textit{Moreover, these commands can be combined by nesting.}} \par
% 在\textbf{}的大括號中放入需要粗體的字。在\textit{}中放入需要斜體的字。在\underline{}中放入需要畫底線的字。同時需要做兩件或三件事時,就把他們一個一個包起來。

The tutorial of how to deal with the words in \LaTeX \ \ ends here, and please go to ``literature.tex" for the next tutorial.

% 如果想新增章節的話就在chapters資料夾裡面新增一個.tex檔,然後以這個文件裡面的前兩行作為開頭就行。文件建立之後到thesis.tex裡面的第112行看怎麼把新的檔案放進論文裡。